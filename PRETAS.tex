\textbf{Fernando António Nogueira Pessoa} (Lisboa, 1888--\textit{id.}, 1935) é o mais
importante poeta português do século~\textsc{xx}. Aos sete anos, muda"-se com a mãe para Durban,
na África do Sul, onde é alfabetizado na língua inglesa.
Em 1905, retorna definitivamente para sua cidade natal e ingressa na Faculdade de Letras
da Universidade de Lisboa. Começa a publicar textos de crítica na revista
\textit{A águia}, em 1912, e a colaborar em jornais e revistas, sendo a principal delas a 
\textit{Orpheu}. Cria os heterônimos Alberto Caeiro, Álvaro
de Campos e Ricardo Reis, o ``semi"-heterônimo'' Bernado Soares e o ortônimo
``Pessoa ele"-mesmo''. Durante sua vida publicou em livro apenas \textit{Mensagem} (1934). 
Trabalhou em Lisboa como tradutor e ``correspondente estrangeiro'' de casas comerciais. 
Falece em decorrência de uma cirrose hepática aos 47 anos, nesta mesma cidade. 

\textbf{Teatro do êxtase} reúne cinco peças de Fernando Pessoa, concebidas 
como poemas dramáticos e destinadas mais à leitura do que à encenação. 
\textit{O~marinheiro} (1915), único drama publicado em vida, foi incluído no
primeiro número da revista \textit{Orpheu} e figura, juntamente com
\textit{Fausto}, como sua peça mais importante.  Definida pelo próprio autor
como um ``drama estático'', a obra de matriz simbolista apresenta o diálogo
entre três mulheres que velam o corpo de uma donzela, sem nenhuma referência
histórica. 
%
\textit{A morte do príncipe} remonta a \textit{Hamlet}, de Shakespeare. 
Trata de um príncipe que
alcança, através de sua viagem delirante pelos arcanos da própria alma,
uma espécie de êxtase visionário, que o leva a afirmar que a única
realidade reside no sonho, isto é, não na própria vida, mas no teatro
da vida. 
%
\textit{Diálogo no jardim do palácio} guarda referências platônicas,
no que diz respeito à reflexão sobre o amor e à dicotomia
entre corpo e alma.
%
\textit{Salomé} insere"-se na rica tradição de leituras do
tema bíblico da \textit{mulher fatal}, ao apresentar o delírio 
da executora de São João Batista diante de sua cabeça decepada. 
%
\textit{Sakyamuni}, por sua vez, representa a ascensão de Siddhartha Gautama ao estado de
iluminação, em que passa a ser reconhecido como Buda. 
%
As peças aqui reunidas são provavelmente as mais acabadas
dentre os muitos fragmentos deixados por seu autor, e apresentam
como eixo comum a concepção pessoana de ``êxtase''.


\textbf{Caio Gagliardi} é professor da Universidade de São Paulo na área de Literatura Portuguesa, onde coordena o grupo de pesquisas Estudos Pessoanos; mestre e doutor em Teoria e História Literária pela \textsc{unicamp}
e pós"-doutor em Teoria Literária pela \textsc{usp}. É autor de \textit{O renascimento do autor: autoria, heteronímia e \textit{fake memoirs}} (Hedra, 2019) e organizador de \textit{Fernando Pessoa \& Cia. não heterônima} (Mundaréu, 2019), entre outras publicações.




